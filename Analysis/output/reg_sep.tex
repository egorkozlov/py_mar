\begin{table}[H]\centering                                  \scriptsize                                 \caption{The average effect of unilateral divorce on the probability that cohabitation ends in breakup vs. marriage. Unit of observation: end of a cohabitation spell}                                   \label{tab:tabsep}                                 \resizebox{0.8\textwidth}{!}{                                 \begin{tabular}{l*{4}{c}} \toprule                             &\multicolumn{4}{c}{\textit{Dependent variable: marry (0), breakup (1)}}\\
\midrule
Unilateral          &     -0.0037         &     -0.0169         &      0.0563         &      0.0486         \\
                    &    (0.0338)         &    (0.0313)         &    (0.0766)         &    (0.0698)         \\
State FE     & Yes & Yes& Yes & Yes\\                       Year cohabitation began FE     & Yes & Yes& Yes & Yes\\                           Additional controls    & No & Yes& No& Yes\\                                           State-specific linear time trend    & No & No & Yes & Yes\\
Dependent variable mean&       0.379         &       0.379         &       0.354         &       0.354         \\
Observations        &        3852         &        3852         &      105384         &      105384         \\
\bottomrule
\noalign{\smallskip}
\end{tabular}
}
\begin{minipage}{\textwidth}
\scriptsize\smallskip
Notes: This table reports the average effect of unilateral divorce on the probability of cohabitation ending in a breakup vs. marriage. The analysis follows the methodology outlined in \cite{borusyak2021} uses and the \textit{cohabitations sample}. The unit of observation is the end of a cohabitation spell (either marriage or breakup) which \textit{started} in state \textit{s} and year \textit{t}. The dummy variable \textit{Unilateral Divorce} takes value 1 if unilateral divorce was in effect in state \textit{s} and year \textit{t}. The additional controls include dummies for ethnicity, age, and education. The sample associated to regressions including State-specific linear time trend excludes five states due to multicollinearity. Standard errors are clustered at the state level. Coefficients that are significantly different from zero are denoted by *10\%, **5\%, and ***1\%.
\\
\end{minipage}
\end{table}
