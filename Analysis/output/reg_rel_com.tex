\begin{table}[H]\centering                                  \scriptsize                                  \caption{The average effect of unilateral divorce on the choice between cohabitation and marriage among newly formed couples, by property regime upon divorce}                    \label{tab:tabrelcom}                                 \resizebox{0.8\textwidth}{!}{                                 \begin{tabular}{l*{4}{c}} \toprule                                 &\multicolumn{4}{c}{\textit{Dependent variable: marry (1), cohabit (0)}}\\                                 \textit{Sample:}& ComP & ComP &  Tit & Tit \\
\midrule
Unilateral          &     -0.1345\sym{***}&     -0.1327\sym{***}&     -0.0637\sym{*}  &     -0.0573\sym{*}  \\
                    &    (0.0507)         &    (0.0451)         &    (0.0332)         &    (0.0325)         \\
State FE     & Yes & Yes& Yes & Yes\\                       Period relationship began FE     & Yes & Yes& Yes & Yes\\                           Additional controls    & No & Yes& No& Yes\\
Dependent variable mean&       0.685         &       0.685         &       0.806         &       0.806         \\
Observations        &        2324         &        2324         &        4631         &        4631         \\
\bottomrule
\noalign{\smallskip}
\end{tabular}
}
\begin{minipage}{\textwidth}
\scriptsize\smallskip
Notes: This table reports the average effect of unilateral divorce on the choice between cohabitation and marriage. The analysis follows the methodology outlined in \cite{borusyak2021} uses and the \textit{relationships sample}. The unit of observation refers to newly formed couples (either married or cohabiting) in state \textit{s} and year \textit{t}. The dummy variable \textit{Unilateral Divorce} takes the value 1 if unilateral divorce was in effect in state \textit{s} and year \textit{t} and 0 otherwise. The additional controls include dummies for ethnicity, age, and education. ComP: community property states; Tit: title based regime. Standard errors are clustered at the state level. Coefficients that are significantly different from zero are denoted by *10\%, **5\%, and ***1\%.
\\
\end{minipage}
\end{table}
