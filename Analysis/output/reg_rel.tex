\begin{table}[H]\centering                                  \scriptsize                                 \caption{The average effect of unilateral divorce on the choice between cohabitation and marriage among newly formed couples}                                   \label{tab:tabrel}                                 \resizebox{0.8\textwidth}{!}{                                 \begin{tabular}{l*{4}{c}} \toprule                                 &\multicolumn{4}{c}{\textit{Dependent variable: marry (1), cohabit (0)} }\\
\midrule
Unilateral          &     -0.0917\sym{***}&     -0.0836\sym{***}&     -0.0846\sym{***}&     -0.0778\sym{***}\\
                    &    (0.0249)         &    (0.0237)         &    (0.0302)         &    (0.0295)         \\
State FE     & Yes & Yes& Yes & Yes\\                       Period relationship began FE     & Yes & Yes& Yes & Yes\\                           Additional controls    & No & Yes& No& Yes\\                                           State-specific linear time trend    & No & No & Yes & Yes\\
Dependent variable mean&       0.725         &       0.725         &       0.725         &       0.725         \\
Observations        &       11165         &       11165         &       11165         &       11165         \\
\bottomrule
\noalign{\smallskip}
\end{tabular}
}
\begin{minipage}{\textwidth}
\scriptsize\smallskip
Notes: This table reports the average effect of unilateral divorce on the choice between cohabitation and marriage. The analysis follows the methodology outlined in \cite{borusyak2021} uses and the \textit{relationships sample}. The unit of observation refers to newly formed couples (either married or cohabiting) in state \textit{s} and year \textit{t}. The dummy variable \textit{Unilateral Divorce} takes the value 1 if unilateral divorce was in effect in state \textit{s} and year \textit{t} and 0 otherwise. The additional controls include dummies for ethnicity, age, and education. Standard errors are clustered at the state level. Coefficients that are significantly different from zero are denoted by *10\%, **5\%, and ***1\%.
\\
\end{minipage}\vspace{-6mm}
\end{table}
