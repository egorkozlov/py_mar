\begin{table}[H]\centering                                  \scriptsize                                                                  \caption{The average effect of unilateral divorce on the choice of cohabitation vs. marriage among newly formed couples. Sample of never movers}                                   \label{tab:tabrelkeep}                                 \resizebox{0.8\textwidth}{!}{                                 \begin{tabular}{l*{4}{c}} \toprule                                 &\multicolumn{4}{c}{\textit{Dependent variable: marry (1), cohabit (0)}}\\
\midrule
Unilateral          &     -0.0928\sym{***}&     -0.0881\sym{***}&     -0.0865\sym{***}&     -0.0842\sym{**} \\
                    &    (0.0278)         &    (0.0273)         &    (0.0322)         &    (0.0335)         \\
State FE     & Yes & Yes& Yes & Yes\\                       Period relationship began FE     & Yes & Yes& Yes & Yes\\                           Additional controls    & No & Yes& No& Yes\\                                           State-specific linear time trend    & No & No & Yes & Yes\\
Dependent variable mean&       0.734         &       0.734         &       0.734         &       0.734         \\
Observations        &        7798         &        7798         &        7798         &        7798         \\
\bottomrule
\noalign{\smallskip}
\end{tabular}
}
\begin{minipage}{\textwidth}
\scriptsize\smallskip
Notes: This table reports the average effect of unilateral divorce on the choice between cohabitation and marriage. The analysis follows the methodology outlined in \cite{borusyak2021} uses and the \textit{relationships sample}. The sample used for this regression includes only respondents who still live in the same state where they were living at sixteen years old. The unit of observation refers to newly formed couples (either married or cohabiting) in state \textit{s} and year \textit{t}. The dummy variable \textit{Unilateral Divorce} takes the value 1 if unilateral divorce was in effect in state \textit{s} and year \textit{t} and 0 otherwise. The additional controls include dummies for ethnicity, age, and education. Standard errors are clustered at the state level. Coefficients that are significantly different from zero are denoted by *10\%, **5\%, and ***1\%.
\\
\end{minipage}
\end{table}
