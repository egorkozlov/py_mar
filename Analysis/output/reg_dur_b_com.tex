\begin{table}[H]\centering                                  \scriptsize                                 \caption{The average effect of unilateral divorce on the probability that a cohabitation spell ends in a breakup by property regime at divorce. Unit of observation: cohabitation-month}                                   \label{tab:tabdurbcom}                                 \resizebox{0.8\textwidth}{!}{                                 \begin{tabular}{l*{4}{c}} \toprule                                 &\multicolumn{4}{c}{\textit{Dependent variable: keep cohabiting (0), breakup (1)}}\\                                 \textit{Sample:}& ComP & ComP &  Tit & Tit \\
\midrule
Unilateral          &     -0.0140\sym{**} &     -0.0144\sym{**} &     -0.0001         &      0.0016         \\
                    &    (0.0059)         &    (0.0071)         &    (0.0028)         &    (0.0036)         \\
State FE     & Yes & Yes& Yes & Yes\\                       Year cohabitation began FE     & Yes & Yes& Yes & Yes\\                           Additional controls    & No & Yes& No& Yes\\
Dependent variable mean&       0.035         &       0.035         &       0.033         &       0.033         \\
Observations        &       25243         &       25243         &       34624         &       34624         \\
\bottomrule
\noalign{\smallskip}
\end{tabular}
}
\begin{minipage}{\textwidth}
\scriptsize\smallskip
Notes: This table reports the average effect of unilateral divorce on the probability of cohabitation ending in a breakup. The analysis follows the methodology outlined in \cite{borusyak2021} uses and the \textit{cohabitations sample}. The unit of observation is one cohabitation-month, which corresponds to a specific month within a particular cohabitation spell. The focus of this table is to report the probability of a cohabitation spell (initiated in state \textit{s} and year \textit{t}) ending in a breakup, with the occurrence of marriage considered as a termination point (right censoring) for the cohabitation period. The dummy variable \textit{Unilateral Divorce} takes value 1 if unilateral divorce was in effect in state \textit{s} and year \textit{t} at the start of the cohabitation spell. The additional controls include dummies for ethnicity, age, education, duration of cohabitation,  and the introduction of unilateral divorce \textit{after} the cohabitation period commenced. ComP: community property states; Tit: title based regime. Standard errors are clustered at the state level. Coefficients that are significantly different from zero are denoted by *10\%, **5\%, and ***1\%.
\\
\end{minipage}
\end{table}
