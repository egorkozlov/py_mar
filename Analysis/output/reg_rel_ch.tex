\begin{table}[H]\centering                                  \scriptsize                                                                  \caption{The average effect of unilateral divorce on the choice between cohabitation and marriage among newly formed couples, by the presence of children}                                   \label{tab:tabrelch}                                 \resizebox{0.8\textwidth}{!}{                                 \begin{tabular}{l*{2}{c}} \toprule                                 &\multicolumn{2}{c}{\textit{Dependent variable: marry (1), cohabit (0)}}\\                                 \textit{Sample:}& Some children &  Childless \\
\midrule
Unilateral          &     -0.0661\sym{***}&     -0.1044\sym{**} \\
                    &    (0.0234)         &    (0.0410)         \\
State FE     & Yes & Yes \\                       Period relationship began FE     & Yes & Yes \\                           Additional controls    & Yes & Yes\\
Dependent variable mean&       0.771         &       0.534         \\
Observations        &        8875         &        2269         \\
\bottomrule
\noalign{\smallskip}
\end{tabular}
}
\begin{minipage}{\textwidth}
\scriptsize\smallskip
Notes: This table reports the average effect of unilateral divorce on the choice between cohabitation and marriage. The analysis follows the methodology outlined in \cite{borusyak2021} uses and the \textit{relationships sample}. The unit of observation refers to newly formed couples (either married or cohabiting) in state \textit{s} and year \textit{t}. The dummy variable \textit{Unilateral Divorce} takes the value 1 if unilateral divorce was in effect in state \textit{s} and year \textit{t} and 0 otherwise. The additional controls include dummies for ethnicity, age, and education. Standard errors are clustered at the state level. Coefficients that are significantly different from zero are denoted by *10\%, **5\%, and ***1\%.
\\
\end{minipage}
\end{table}
